\documentclass[12pt]{article}
\usepackage[a4paper,margin=1in]{geometry}
\usepackage{amsmath,amssymb}
\usepackage{xcolor}
\usepackage{hyperref}
\usepackage{listings}

% ---------------- Listings setup ----------------
\definecolor{codebg}{rgb}{0.97,0.97,0.97}
\lstset{
  language=Python,
  basicstyle=\ttfamily\small,
  backgroundcolor=\color{codebg},
  frame=single,
  framerule=0.4pt,
  numbers=left,
  numberstyle=\tiny,
  stepnumber=1,
  tabsize=4,
  breaklines=true,
  showstringspaces=false,
  keywordstyle=\color{blue!70!black},
  stringstyle=\color{red!60!black},
  commentstyle=\color{green!40!black}
}

\title{Linear Regression From Scratch: Mathematics and Implementation}
\author{(Soorya S)}
\date{\today}

\begin{document}
\maketitle

\section{Feature Matrix}
We define the feature matrix $X\in\mathbb{R}^{m\times n}$ as
\[
X=
\begin{bmatrix}
x_{11} & x_{12} & \dots & x_{1n}\\
x_{21} & x_{22} & \dots & x_{2n}\\
\vdots & \vdots & \ddots & \vdots\\
x_{m1} & x_{m2} & \dots & x_{mn}
\end{bmatrix},
\qquad
\vec x^{(i)}=(x_{i1},x_{i2},\dots,x_{in}).
\]
Each row of $X$ is a feature vector $\vec x^{(i)}$.

\section{Prediction Rule}
For a sample $\vec x^{(i)}$, the prediction is
\[
\hat y^{(i)}=f(\vec x^{(i)})=\vec w\cdot\vec x^{(i)}+b,
\]
so vectorized:
\[
\hat{\vec y}=X\vec w + b\,\mathbf 1.
\]

\section{Error Vector}
For each sample,
\[
e^{(i)}=\hat y^{(i)}-y^{(i)},
\]
and the stacked error vector is
\[
\vec e=
\begin{bmatrix}
\hat y^{(1)}-y^{(1)}\\
\hat y^{(2)}-y^{(2)}\\
\vdots\\
\hat y^{(m)}-y^{(m)}
\end{bmatrix}
\in\mathbb{R}^m.
\]

\section{Gradient of the Multivariable Cost}
We use the mean squared error with the $\tfrac12$ convention:
\[
J(\vec w,b)=\frac{1}{2m}\sum_{i=1}^m\!\big(\hat y^{(i)}-y^{(i)}\big)^2
=\frac{1}{2m}\|\vec e\|_2^2.
\]
Its gradients are
\[
\nabla_{\vec w}J=\frac{1}{m}X^\top\vec e,
\qquad
\frac{\partial J}{\partial b}=\frac{1}{m}\sum_{i=1}^m e^{(i)}.
\]

\section{Parameter Updates (Stick to Notes)}
With learning rate $\alpha$:
\[
\vec w \leftarrow \vec w - \alpha\,\nabla_{\vec w}J
=\vec w-\alpha\cdot\frac{1}{m}X^\top\vec e,
\qquad
b \leftarrow b - \alpha\,\frac{\partial J}{\partial b}
=b-\frac{\alpha}{m}\sum_{i=1}^m e^{(i)}.
\]

\section{Normalization (z-Score)}
Before regression we normalize each feature using training statistics:
\[
x'^{(i)}_j=\frac{x^{(i)}_j-\mu_j}{\sigma_j},
\quad
\mu_j=\frac{1}{m}\sum_{i=1}^m x^{(i)}_j,
\quad
\sigma_j=\sqrt{\frac{1}{m}\sum_{i=1}^m (x^{(i)}_j-\mu_j)^2},
\]
and if $\sigma_j=0$ we set $\sigma_j\leftarrow 1$.

\section{Dataset Split (Stick to Notes)}
We split the dataset as:
\[
\text{Train: }80\%,\quad \text{Validation: }10\%,\quad \text{Test: }10\%.
\]

% ===================== CODE SNIPPETS =====================

\section{Implementation Snippets (Exact Code)}
\subsection*{Loading Data and Building $X,\vec y$}
\begin{lstlisting}
import numpy as np
import pandas as pd

df = pd.read_csv("housing.csv", delim_whitespace=True, header=None)
X = df.iloc[:, :-1].to_numpy(dtype=float)
y = df.iloc[:, -1].to_numpy(dtype=float)
\end{lstlisting}

\subsection*{z-Score Normalization (with zero-std guard)}
\begin{lstlisting}
m, n = X.shape
mu = X.mean(axis=0)
sigma = X.std(axis=0, ddof=0)
sigma[sigma == 0] = 1.0
X = (X - mu) / sigma
\end{lstlisting}

\subsection*{Parameters and Prediction Function}
\begin{lstlisting}
w = np.zeros(n, dtype=float)
b = 0.0

def f(X, w, b):
    # Vectorized: \hat{y} = X w + b 1
    return X @ w + b
\end{lstlisting}

\subsection*{Gradient Step (Implements \(X^\top e\) and bias average)}
\begin{lstlisting}
def step(X, y, w, b, alpha):
    m = len(y)
    e = f(X, w, b) - y            # e = \hat{y} - y
    grad_w = (X.T @ e) / m        # \nabla_w J = (1/m) X^T e
    grad_b = e.mean()             # dJ/db = (1/m) sum e_i
    return w - alpha * grad_w, b - alpha * grad_b
\end{lstlisting}

\subsection*{Cost Function \(J(\vec w,b)=\tfrac{1}{2m}\|e\|^2\)}
\begin{lstlisting}
def cost(X, y, w, b):
    e = f(X, w, b) - y
    return 0.5 * np.mean(e ** 2)
\end{lstlisting}

\subsection*{Batch Gradient Descent with Tolerance}
\begin{lstlisting}
alpha = 0.05
max_epochs = 5000
tol = 1e-8
prev_J = None

for epoch in range(1, max_epochs + 1):
    w, b = step(X, y, w, b, alpha)
    J = cost(X, y, w, b)
    if prev_J is not None and abs(prev_J - J) < tol:
        break
    prev_J = J
\end{lstlisting}

\subsection*{Predictions and Metrics}
\begin{lstlisting}
y_pred = f(X, w, b)
mse = np.mean((y - y_pred) ** 2)

print(f"final_cost J(w,b): {J:.6f}")
print(f"MSE: {mse:.6f}")
print("bias b:", round(b, 6))
print("weights w:", np.round(w, 6))
\end{lstlisting}

\end{document}